\clearpage
\section{Rychlokurz LaTeXu}

\subsection{\LaTeX}

\LaTeX{} je sázecí systém pro tvorbu odborných dokumentů.
Uživatel píše \emph{plain text}, který je pak dle použitých značek převeden na výsledný dokument.
Oproti tomu například ve Wordu píšmene formátovaný text, jinak také \emph{\enquote{What You See Is What You Get}}.
Hlavními výhodami \LaTeX{}u jsou především vysoká typografická kvalita, jednoduchá přenositelnost (všichni známe rozbité Word dokumenty) a jednotnost stylu (velikosti písma, typy fontů) -- vše je definováno v šabloně.

\subsection{LaTeX na FEKTu}

Všechny oficiální šablony pro studium na FEKTu můžete najít na 
\url{https://latex.feec.vutbr.cz/}.

\subsection{Jak na Overleaf?}

Nejdříve si z \url{https://github.com/VUT-FEKT-IBE/FEKT.tex}
stáhneme tuto šablonu.
Dále si na overleafu vybereme možnost \texttt{New Project > Upload Project} a vybereme staženou šablonu.

Po otevření šablony najedeme na menu v levém horním rohu v možností \texttt{Main Document} a vybereme \texttt{main.tex} nebo \texttt{text/01.tex}.
Dokument samotný pak kompilujeme \textbf{mimo} \texttt{shared.tex}.
Jinak se PDF nevytvoří.

Pokud chcete oddělit jednotlivé textové soubory a vkládat je do dokumentu zvlášť, stačí v záložce \texttt{text} vytvořit nový soubor s příponou \texttt{.tex}.
V souboru \texttt{main.tex} pak daný soubor načteme pomocí příkazu \verb|\include{file_path}|.

Dokument samotný kompilujeme pomocí tlačítka \texttt{Recompile}, případně pomocí zkratky CTRL+ENTER.
Tlačítko pro stažení PDF souboru se nachází vedle \texttt{Recompile}.


\clearpage
\section[Struktura LaTeXového ]{Struktura \LaTeX{}ového textu}

Pro~vytvoření \enquote{hlavního nadpisu} (jako je název kapitoly hned v~předcházejícím řádku) použijeme příkaz \verb|\section{text}|.
Jednotlivé oddíly se číslují automaticky vzesupně.

Pro~vytvoření podnadipsů (třeba část \ref{sec:obrazky} níže) použijeme příkaz  \verb|\subsection{text}|.
Stejně jako hlavní nadpisy se čísluje automaticky.

Nové odstavce se vytváří automaticky, pokud vynecháme jeden řádek.
Řádek jako takový můžeme také zalomit dvěma zpětnými lomítky \verb|\\|.

\begin{figure}[ht]
\onehalfspacing
\begin{mdframed}
\begin{verbatim}
V~půlce této věty řádek lámu pouze v~kódu
a~text je přesto sázen dál.
\end{verbatim}

V~půlce této věty řádek lámu pouze v~kódu
a~text je přesto sázen dál.

\begin{verbatim}
V~půlce této věty řádek lámu použitím\verb|\\|\\
a~text tak není sázen až do~konce, i~když pro~to prostor má.
\end{verbatim}

V~půlce této věty řádek lámu použitím\verb|\\|\\
a~text tak není sázen až do~konce, i~když pro~to prostor má.
\end{mdframed}
\end{figure}

Nezlomitelná mezera se píše pomocí vlnovky (\texttt{\~}): \verb|v~domě|.
Používejte ji před všemi předložkami a~spojkami, ty na~konce řádků nepatří.

Psát ji vždy a~všude (a~ne jen když se zrovna dostane na~konec řádku z~důvodu sazby) má několik velkých výhod.
Především vám to ušetří čas: nebudete si při psaní textu hlídat jestli náhodou na~konci řádku nejste a~můžete se soustředit pouze na~obsah.
A~ušetříte tím i~čas, protože tyto tvrdé mezery nebudete muset zpětně doplňovat.

\LaTeX{}ové soubory můžete komentovat pomocí značky \verb|%|: cokoliv za~touto značkou se sázet nebude (a~textové editory podporující \LaTeX{}ovou syntaxi vám takový text barevně zvýrazní).
Můžete tak v~textu zanechávat poznámky (své zdroje, nejasnosti, TODO) aniž by byly vidět ve~finální verzi dokumentu.

\subsection{Seznamy}

Nečíslované seznamy se vytváří v~prostředí \verb|itemize|:
jednotlivé body oddělujeme značkou \verb|\item|.

Číslované seznamy fungují obdobně a~vytváří se prostředím \verb|enumerate|.
Tato dvě prostředí do~sebe lze vnořovat.

\begin{figure}[ht]
\begin{mdframed}
\onehalfspacing
\begin{verbatim}
\begin{itemize}
  \item itemize prvni urovne
  \begin{itemize}
    \item itemize druhe urovne
    \begin{enumerate}
      \item prvni polozka enumerate
      \item druha polozka enumerate
    \end{enumerate}
  \end{itemize}
\end{itemize}
\end{verbatim}

\vspace*{1em}

\begin{itemize}
    \item itemize první úrovně
    \begin{itemize}
        \item itemize druhé úrovně
        \begin{enumerate}
            \item první položka enumerate
            \item druhá položka enumerate
        \end{enumerate}
    \end{itemize}
\end{itemize}
\end{mdframed}
\end{figure}
\FloatBarrier

Většina autorů píše stejným způsobem jako ve~Wordu, a~\enquote{Enter} v~kódu píše až na~konci odstavce.
Psát každou větu na~svůj řádek však přináší výhodu: když je kód verzován pomocí \texttt{git}u, vznikají menší \texttt{diff}y a~změny jsou lépe vidět.
Tak je strukturován i~tento dokument.
 
\subsection{Vizuální značky}

\begin{table}[ht]
\centering
\begin{tabular}{|l|l|}
značka & ukázka \\
\hline \hline
\texttt{textbf} & \textbf{tučné písmo} \\
\texttt{textit} & \emph{kurzíva} \\
\texttt{enquote} & \enquote{text v~uvozovkách} \\
\texttt{texttt} & \texttt{strojový text} \\
\end{tabular}
\end{table}

\subsection{Obrázky}
\label{sec:obrazky}

Do~složky \texttt{images/} nahrajte obrázek ve~formátu \texttt{jpg}, \texttt{png} nebo \texttt{pdf}.

Pro~jeho vložení do~textu se používá prostředí \texttt{figure}, samotné vložení zajištuje příkaz \texttt{includegraphics}.
Nemusíte psát příponu, \LaTeX{} vybere za~vás tu nejlepší možnost.

Je zvykem také uvádět popis obrázku (\texttt{caption}).
Všechny obrázky jsou automaticky číslované; pokud pod~popisem uvedete i~\texttt{label}, můžete se na~číslo odkazovat v~textu%
\footnote{Což platí pro~všechny důležité objekty v~\LaTeX{}u; takto v~textu výše odkazujeme na~aktuální kapitolu.}%
.

\begin{figure}[ht]
\begin{mdframed}
\onehalfspacing
\begin{verbatim}
\begin{figure}[ht]
    \onehalfspacing
    \centering
    \includegraphics{cesta/k/souboru} 
    \caption{Popis obrázku}
    \label{fig:label}
\end{figure}
\end{verbatim}
\end{mdframed}
\end{figure}
\FloatBarrier

\subsection{Matematika a~vzorce}

Matematika může být sázena ve~dvou režimech.

Režim \emph{inline} text sází do~řádku jako text a~je vhodný pro~menší nebo méně důležité vzorce:
$\sum_{i=0}^{10} \sin (x + \pi)$.
Text se zde uzavírá mezi značky \verb|$|.

Režim \emph{outline} lze použít naprosto stejně, ale vzorec uzavřete mezi dvojité \verb|$$|:
$$\sum_{i=0}^{10} \sin (x + \pi).$$

Pokud chcete vzorce číslovat nebo jich vypsat více za~sebou, lze použít prostředí \texttt{align}:
% Zdroj rovnic: https://latex-tutorial.com/align-equations/
\begin{align*}
f(u) & =\sum_{j=1}^{n} x_jf(u_j) \\
     & =\sum_{j=1}^{n} x_j \sum_{i=1}^{m} a_{ij}v_i \\
     & =\sum_{j=1}^{n} \sum_{i=1}^{m} a_{ij}x_jv_i
\end{align*}

\begin{table}[ht]
\centering
\begin{tabular}{|l|l|c|}
značka & význam & ukázka \\
\hline \hline
\texttt{\_} & dolní index & $x_1$ \\
\texttt{\^} & horní index & $x^2$ \\
\verb|^{}|  & více než jeden znak v~indexu & $x^{e-1}$ \\
\verb|\{|, \verb|\}| & složené závorky & $\{1, 2, 3\}$ \\
\verb|\langle|, \verb|\rangle| & ostré závorky & $\langle 1, 3 \rangle$ \\
\verb|\left(|, \verb|\right)| & závorka přizpůsobená obsahu & $\left(\begin{matrix}1\\2\end{matrix}\right)$
\end{tabular}
\end{table}

\begin{table}[ht]
\centering
\begin{tabular}{|l|l|c|}
příkaz & význam & ukázka \\
\hline \hline
\verb|\mathrm{}| & režim textu uvnitř vzorce & $\int_1^5 e^x \mathrm{d}x$ \\
\verb|\frac{}{}| & zlomek & $\frac{1}{2}$ \\
\verb|\sum_{}^{} {}| & suma s~dolní a~horní hranicí & $\sum_{i=1}^{2} 2i$ \\
\verb|\int_{}^{} {}| & integrál s~dolní a~horní hranicí & $\int_{1}^{2} 2x \, \mathrm{d}x$ \\
\verb|\begin{matrix}| & matice & $\left[ \begin{matrix}
1 & 2 \\
3 & 4 \\
\end{matrix} \right]$ \\
\end{tabular}
\end{table}

Písmena řecké abecedy se sází jejich názvy: $\alpha$ \verb|\alpha|, $\beta$ \verb|\beta|, \dots, $\Omega$ \verb|\Omega|.

\begin{figure}[ht]
\begin{mdframed}
\onehalfspacing
\begin{verbatim}
$E \left\{ \chi^2(n) \right\} = E(U_1^2 + \cdots + U_n^2) = n$
\end{verbatim}

$E \left\{ \chi^2(n) \right\} = E(U_1^2 + \cdots + U_n^2) = n$
\end{mdframed}
\end{figure}
\FloatBarrier

\subsection{Tabulky}

Tabulky se sází prostředím \texttt{table} a \texttt{tabular}%
\footnote{
    Existuje více implementací podobných \texttt{tabular}, například \texttt{tabularx}.
    Chvíli je to matoucí, jestli je nejprve \texttt{table} nebo \texttt{tabular}, ale na to si rychle zvyknete.
}%
.
Stejně jako u~obrázků se v~něm používají příkazy \texttt{caption} nebo \texttt{label}.
V~česky sázeném dokumentu se dle normy ISO sází nadpisy tablek \emph{pod} obrázky a~grafy, ale \emph{nad} tabulky.

Prostředí \texttt{tabular} má povinný parametr kterým definujete počet, zarovnání a~oddělení sloupců: \verb_{|l|c|c|r|}_ je značka pro~čtyři sloupce zarovnané vlevo, do~středu, do~středu a~vpravo, s~ohraničením.
Uvnitř tabulky se konec řádku značí užitím \verb|\\| (stejně jako zalomení řádku v~textu), pole tabulky se oddělují znakem \verb|&|.

\begin{mdframed}
\begin{verbatim}
\begin{table}[ht]
    \onehalfspacing
    \centering
    \begin{tabular}{c|c|c}
        num 1 & num 2 & num 3 \\
        \hline\hline
        1 & 1 & 1 \\
        2 & 2 & 2 \\
        3 & 3 & 3 \\
        4 & 4 & 4 \\ 
    \end{tabular}
\end{table}
\end{verbatim}
    \label{tbl:dvanact_hodnot}
    \begin{tabular}{c|c|c}
        num 1 & num 2 & num 3 \\
        \hline\hline
        1 & 1 & 1 \\
        2 & 2 & 2 \\
        3 & 3 & 3 \\
        4 & 4 & 4 \\ 
    \end{tabular}
\end{mdframed}
\FloatBarrier
